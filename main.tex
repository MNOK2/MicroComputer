\documentclass[a4j, dvipdfmx]{jsarticle}

\usepackage{graphicx, color}
\usepackage{jvlisting, listings}

\lstset{
 	language = C,
 	backgroundcolor={\color[rgb]{1.0, 1.0, 1.0}},
 	breaklines = true,	
 	breakindent = 10pt,
 	basicstyle = \ttfamily\scriptsize,
 	commentstyle = \itshape\color[rgb]{0.5 , 0.5, 0.5},
 	classoffset = 0,
 	keywordstyle = \bfseries\color[rgb]{1.0, 0.0, 0.0},
 	stringstyle = \ttfamily\color[rgb]{0.0, 0.5, 0.0},
 	frame = TBrl,
 	framesep = 5pt,
 	numbers = left,
 	stepnumber = 1,
 	numberstyle = \tiny,
 	tabsize = 4,
 	captionpos = t
}

\begin{document}

\section{ソースコード}
ソースコード\ref{src:homework}に作成したコードを示す。
\lstinputlisting[caption = 課題, label = src:homework]{src/homework.c}

\section{工夫した点}
ソースコード\ref{src:homework}にて工夫した点を次にまとめる。
\begin{itemize}
	\item continue文の活用により、while文内のネストを殲滅した。
	\item 状態遷移の考え方を導入することにより、わかりやすいコードを書けた。
	\item 表示の更新を行うまでの0.5秒間に押されるスイッチの検知を可能とした。
\end{itemize}

\end{document}